\documentclass{article}
\usepackage{amsmath}

\newtheorem{ring}{Definition}
\newtheorem{group}{Definition}
\newtheorem{field}{Definition}

\begin{document}

\begin{ring}

\end{ring}

\begin{group}
A group is a set, $G$, with an operation, $\cdot$, that combines
any two elements $a$ and $b$ to form another element, written as
$a \cdot b$ or $ab$. In order to qualify as a group, the set
and the operation, $(G, \cot)$, must satisfy the \textit{group axioms}
which are as follows.

\textbf{Closure}: For all $a, b \in G$, the result of the operation,
$a \cdot b$, is also in $G$.

\textbf{Associativity}: For all $a, b \text{and} c \in G, (a \cdot b)
\cdot c = a \cdot (b \cdot c)$.

\textbf{Identity element}: There exists an element $e \in G$, such that
for every element $a \in G, e \cdot a = a \cdot e = a$.

\textbf{Inverse element}: For each $a \in G$, there exists an element
$b \in G$ such that $a \cdot b = b \cdot a = e$, where $e$ is the
identity element.

\end{group}

\begin{field}

\end{field}

\end{document}



