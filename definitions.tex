\documentclass{article}
\usepackage{amsmath}


\begin{document}


A \textit{ring} a set $R$ with two binary operations + and $\cdot$, called
addition and multiplication, which maps every pair of elements of $R$
to a unique element of $R$. In order to qualify as a ring, these operations
must satisfy the \textit{ring axioms}, which must be true for all
$a, b, c \in R$. The ring axioms are:

\begin{enumerate}
  \item $(a + b) + c = a + ( b + c)$ (+ is associative)
  \item There is an element $0 \in R$ such that $0 + a = a$ (0 is the \textit{zero element}
  \item $a + b = b + a$ (+ is commutative)
  \item For each $a \in R$ there exists $-a \in R$ such that $a + (-a) = (-a) + a = 0$ ($-a$ is the inverse element of $a$)
  \item $(a \cdot b) \cdot c = a \cdot (b \cdot c)$ ($\cdot$ is associative)
  \item $a \cdot (b + c) = (a \cdot b) + (a \cdot c)$ (left distributivity)
  \item $(b + c) \cdot a = (b \cdot a) + (c \cdot a)$ (right distributivity)
  \item There is an element $1 \in R$ such that $a \cdot 1 = 1 \cdot a = a$ (multiplicative identity)
\end{enumerate}

A \textit{group} is a set, $G$, with an operation, $\cdot$, that combines
any two elements $a$ and $b$ to form another element, written as
$a \cdot b$ or $ab$. In order to qualify as a group, the set
and the operation, $(G, \cot)$, must satisfy the \textit{group axioms}
which are as follows.

\textbf{Closure}: For all $a, b \in G$, the result of the operation,
$a \cdot b$, is also in $G$.

\textbf{Associativity}: For all $a, b \text{ and } c \in G, (a \cdot b)
\cdot c = a \cdot (b \cdot c)$.

\textbf{Identity element}: There exists an element $e \in G$, such that
for every element $a \in G, e \cdot a = a \cdot e = a$.

\textbf{Inverse element}: For each $a \in G$, there exists an element
$b \in G$ such that $a \cdot b = b \cdot a = e$, where $e$ is the
identity element.
\newline

A \textit{field} is a set $F$ with two operations, addition and multiplication,
such that the following axioms hold:

\textbf{Closure}: For all $a, b \in F$ both $a + b \text{ and } a \cdot b$
are in $F$

\textbf{Associativity}: For all $a, b, \text{ and } c \in F$ the following
equalities hold: $a + (b + c) = (a + b) + c \text{ and } a \cdot (b \cdot c) =
(a \cdot b) \cdot c$

\textbf{Commutativity}: For all $a \text{ and } b \in F$ the following 
equalities hold: $a + b = b + a$ and $a \cdot b = b \cdot a$

\textbf{Existence of Identity Elements}: There exists an element of $F$,
called the \textit{additive identity} and denoted by 0, such that for all
$a \in F, a + 0 = a$. Likewise, there is an element, called the
\textit{multiplicative identity} and denoted by 1, such that for all $a \in F,
a \cdot b = a$. The additive and multiplicative identities are required
to be distinct.

\textbf{Existence of Inverse Elements}: For every $a$ in $F$, there exists
an element $-a$ in $F$, such that $a + (-a) = 0$. Similarly, for any $a$ in $F$,
other than 0, there exists an element $a^{-1}$ in $F$ such that $a \cdot a^{-1} = 1$

\textbf{Distributivity}: For all $a, b, \text{ and } c \in F$ the following equality
holds: $a \cdot (b + c) = (a \cdot b) + (a \cdot c)$


\end{document}



