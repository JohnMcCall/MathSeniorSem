\documentclass{article}
\usepackage{amsmath}

\author{John McCall}
\title{Senior Seminar Proposal: Error Correcting Codes}

\begin{document}

\maketitle

\section{Introduction}
As technology progresses, we rely much more on electronic means of communication.
This communication comes in many forms, such as sending emails or playing music 
from a CD. However, this communication is not perfect and some of the data
may be lost or corrupted in transmission. There are many potential sources
of this corruption, for instance a CD may become scratched. It is an unrealistic
goal to try to prevent all corruption from occurring.

For this reason error correction codes have been developed. Error correction
codes are designed to fix corruptions rather than prevent them. A CD with
minor scratches can still play music perfectly. This is because of
an error correction code, which extrapolates the data corrupted by the
scratch based on the uncorrupted data.

One specific error correction code is the Reed-Solomon code. Reed-Solomon codes
are used in many applications. From consumer electronics such as CDs and DVDs,
to deep-space transmissions on the Voyager space probe. The goal of this
senior seminar is to understand how and why Reed-Solomon codes work. 

\section{Current Work}

\section{Future Plans}


\end{document}
