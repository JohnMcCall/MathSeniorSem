\documentclass{beamer}

\mode<presentation>
{
  \usetheme{CambridgeUS}
  \setbeamercovered{transparent}
}

\usepackage{amsmath}
\usepackage{verbatim}

% The text in square brackets is the short version of your title and will be used in the
% header/footer depending on your theme.
\title[Generalized Reed-Solomon Codes]{Generalized Reed-Solomon Codes}

% Sub-titles are optional - uncomment and edit the next line if you want one.
% \subtitle{Why does sub-tree crossover work?}

% The text in square brackets is the short version of your name(s) and will be used in the
% header/footer depending on your theme.
\author[McCall]{John McCall}
\institute[U of Minn, Morris]
{
  Division of Science and Mathematics \\
  University of Minnesota, Morris \\
  Morris, Minnesota, USA
}

\AtBeginSection[]
{
  \begin{frame}<beamer>
    \frametitle{Outline}
    \tableofcontents[currentsection, hideothersubsections]
  \end{frame}
}

\begin{document}

\begin{frame}
	\titlepage
\end{frame}

\section*{Overview}

\begin{frame}
	\frametitle{Overview}
	\begin{itemize}
		\item Data transmission has become vital in today's society.
		
		\item Transmission methods are not perfect.
		
		\item Error correcting codes help to alleviate the burden of transmission.
	\end{itemize}
	
\end{frame}

\begin{frame}
	\frametitle{Outline}
	\tableofcontents[hideallsubsections]
\end{frame}

\section{Defining our Field}

\begin{frame}
	\frametitle{Introducing $\mathbb{F}_{64}$}
	The finite field of 64 elements.
	
	[Insert picture of ASCII table showing the elements]
\end{frame}

\begin{frame}
	\frametitle{Representation of our Elements}
	\begin{itemize}
		\item Each element is encoded as a 6-bit binary string.
		\item We can represent that string as a polynomial in $\mathbb{F}_{2}[x]$.
	\end{itemize}
	
	For example:
	\begin{itemize}
		 \item \# $\rightarrow$ 000011 $\rightarrow x + 1$ 
	\end{itemize}
\end{frame}

\begin{frame}
	\frametitle{Addition in $\mathbb{F}_{64}$}
	\begin{itemize}
		\item Regular polynomial addition, but the coefficients are added modulo 2.
		\item $(x^{4} + x^{2} + x + 1) + (x^{3} + x + 1) = x^{4} + x^{3} + x^{2}$.
	\end{itemize}
\end{frame}

\begin{frame}
	\frametitle{Multiplication in $\mathbb{F}_{64}$}
	\begin{itemize}
		\item Similar to regular multiplication, except the result cannot be larger than degree 5.
		\item We have to use an irreducible polynomial to mod out terms that are of too high a degree.
		\item The polynomial: $x^{6} + x + 1 \rightarrow x^{6} = x + 1$.
	\end{itemize}
\end{frame}

\begin{frame}
	\frametitle{Multiplication Example}
	\begin{itemize}
		\item $(x^{5} + x^{2} + x + 1)(x^{4} + x)$
		\item $= x^{9} + x^{6} + x^{6} + x^{5} + x^{4} + x^{3} + x^{2} + x$
		\item $= x^{9} + x^{5} + x^{4} + x^{3} + x^{2} + x$
		\item $= (x^{6})(x^{3}) + x^{5} + x^{4} + x^{3} + x^{2} + x$
		\item $= (x + 1)(x^{3}) + x^{5} + x^{4} + x^{3} + x^{2} + x$
		\item $= x^{4} + x^{3} + x^{5} + x^{4} + x^{3} + x^{2} + x$
		\item $= x^{5} + x^{2} + x$
	\end{itemize}
\end{frame}

\begin{frame}
	\frametitle{Multiplication Table}
	[Insert Multiplication Table Here]
\end{frame}

\section{Codes}

\begin{frame}
	\frametitle{Introduction to Codes}
	A \textit{code} is a rule for converting information from one representation into another.
	
	\begin{itemize}
		\item $A \rightarrow 0$
		\item $B \rightarrow 1$
		\item $ \vdots $
		\item $Z \rightarrow 25$
	\end{itemize}
\end{frame}

\begin{frame}
	\frametitle{Encoding}
	\textit{Encoding} is the act of conversion, following the rules of the code.
	
	\begin{itemize}
		\item HELLO $\rightarrow$ 8 5 12 12 15
	\end{itemize}
\end{frame}

\begin{frame}
	\frametitle{Errors}
	\textit{Random errors} are a type of error that corrupts individual symbols during transmission.
	\begin{itemize}
		\item 8 5 1\textbf{2} 12 15 $\rightarrow$ 8 5 1\textbf{9} 12 15
	\end{itemize}
	
	\textit{Burst errors} are errors that corrupt a large chunk of symbols.
	\begin{itemize}
		\item 8 5 \textbf{12 12 1}5 $\rightarrow$ 8 5 \textbf{19 24 }5
	\end{itemize}
\end{frame}

\begin{frame}
	\frametitle{Error Correction}
	\textit{Error detecting codes} are a type of code that can detect when these errors occur.
	
	\textit{Error correcting codes} are a type of code that can correct these errors.
\end{frame}

\subsection{Cyclic Redundancy Check}

\begin{frame}
	\frametitle{Introduction to CRC}
	\begin{itemize}
		\item Used to detect accidental changes in data.
		\item Appends a check value to the message prior to transmission.
		\item After transmission the check value is recomputed and compared to the original value.
	\end{itemize}
\end{frame}

\begin{frame}
	\frametitle{Application and Integrity}
	\begin{itemize}
		\item CRC is good at detecting random errors and burst errors.
		\item It is not suitable for detecting intentional modifications to the data.
	\end{itemize}
\end{frame}

\begin{frame}
	\frametitle{Example}
	To compute a binary CRC with a 3-bit check value:
	\begin{itemize}
		\item Start with our message encoded in binary: 11010011101100
		\item Find a irreducible polynomial of degree 3: $x^{3} + x + 1 \rightarrow$ 1011
	\end{itemize}
\end{frame}

\begin{frame}
	\frametitle{Coming soon:}
	The remainder of the CRC example.
\end{frame}

%\subsection{Linear Codes}

%\begin{frame}
%	\frametitle{Linear Codes}
%\end{frame}

\section{Generalized Reed-Solomon Codes}

\subsection{Reed-Solomon Codes}

\begin{frame}
	\frametitle{History}
	\begin{itemize}
		\item Introduced in 1960 by I.S. Reed and G. Solomon.
		\item Useful in practical applications and mathematically interesting.
		\item Used CD players and deep-space communications.
	\end{itemize}
\end{frame}

\begin{frame}
	\frametitle{The Code}
	\begin{itemize}
		\item Reed-Solomon (RS) codes are a liner, error correcting code.
		\item A RS-code is capable of correcting $\frac{n - k}{2}$ symbol errors. Where $n$ is the total number of symbols, and $k$ is the number of data symbols.
		\item Great at correcting burst errors.
	\end{itemize}
\end{frame}

\begin{frame}
	\frametitle{Representation}
	A message encoded using an RS-code is a polynomial with the message symbols embedded in the coefficients.
	
	For our example, the polynomial would be in in $\mathbb{F}_{64}[x]$ and is our codeword.
	\begin{itemize}
		\item $\text{H} + \text{E}x + \text{L}x^{2} + \text{L}x^{3} + \text{O}x^{4}$.
	\end{itemize}	
	

\end{frame}

\subsection{Generalized Reed-Solomon Codes}

\begin{frame}
	\frametitle{Generalized Reed-Solomon Codes}
	\begin{itemize}
		\item There is an alternate representation, known as Generalized Reed-Solomon (GRS) Codes.
		\item In this representation the message polynomial is evaluated at $n$ distinct points and multiplied by a scalar. 
		\item The result is a $n$-dimensional vector, which is used as the codeword.
		\item This is the representation that will be used for the remainder of the presentation.
	\end{itemize}
	
\end{frame}

\begin{frame}
	\frametitle{The Code}
	Let $F$ be a field. Choose nonzero elements $\hat{v} = v_{1}, v_{2}...,v_{n} \in F$ and distinct elements $\hat{\alpha} = \alpha_{1}, \alpha_{2}...,\alpha_{n} \in F$.\\~\\
	
	$C = $ GRS $_{n, k}(\hat{\alpha}, \hat{v}) = \{(v_{1}f(\alpha_{1}),v_{2}f(\alpha_{2}),...,v_{n}f(\alpha_{n}) | f(x) \in F[x]_{k}\}$
	\begin{itemize}
		\item $F = \mathbb{F}_{64}$.
		\item $\hat{v} = (!, !, ..., !)$.
		\item $\hat{\alpha} = (A, B, ..., W)$.
		\item $f(x)$ is our \textit{message polynomial}.
	\end{itemize}
\end{frame}

\begin{frame}
	\frametitle{Message Polynomial}
	\begin{itemize}
		\item The message: "THIS IS MAJOR TOM".
		\item $f(x) = \text{T} + \text{H}x + \text{I}x^{2} + \text{S}x^{3} + \text{ }x^{4} + \text{I}x^{5} + \text{S}x^{6} + \text{ }x^{7} + \text{M}x^{8} + \text{A}x^{9} + \text{J}x^{10} + \text{O}x^{11} + \text{R}x^{12} + \text{ }x^{13} + \text{T}x^{14} + \text{O}x^{15} + \text{M}x^{16}$
		\item $\hat{c} = \hat{p} + \hat{e}$
	\end{itemize}
\end{frame}

\subsection{Encoding GRS Codes}

\begin{frame}
	\frametitle{The Codeword}
	The codeword, $\hat{c} = (v_{1}f(\alpha_{1}),v_{2}f(\alpha_{2}),...,v_{n}f(\alpha_{n}))$
	\begin{itemize}
		\item $\hat{c} = (!f(\text{A}),!f(\text{B}),...,!f(\text{W}))$
		\item $\hat{c} = \{\text{T}, \text{(}, \text{U}, \text{8}, \text{P}, \text{K}, \text{G}, \text{N}, \text{W}, \text{P}, \text{4}, \text{K}, \text{'}, \text{\_}, \text{N}, \text{(}, \text{M}, \text{H}, \text{K}, \text{1}, \text{"}, <, \text{K}\}$
	\end{itemize}
	
\end{frame}

\subsection{Decoding GRS Codes}

\begin{frame}
	\frametitle{The Received Message}
	\begin{itemize}
		\item $\hat{c} = \{\text{T}, \text{(}, \text{U}, \text{8}, \text{P}, \text{K}, \textbf{G}, \text{N}, \text{W}, \text{P}, \textbf{4}, \text{K}, \text{'}, \text{\_}, \text{N}, \text{(}, \text{M}, \text{H}, \text{K}, \text{1}, \text{"}, <, \text{K}\}$
		\item $\hat{p} = \{\text{T}, \text{(}, \text{U}, \text{8}, \text{P}, \text{K}, \textbf{P}, \text{N}, \text{W}, \text{P}, \textbf{\&}, \text{K}, \text{'}, \text{\_}, \text{N}, \text{(}, \text{M}, \text{H}, \text{K}, \text{1}, \text{"}, \textbf{R}, \text{K}\}$
	\end{itemize}
\end{frame}

\begin{frame}
	\frametitle{Finding the Dual Code}
	\begin{itemize}
		\item Want to find $\hat{u}$ such that $C^{\perp} = $ GRS $_{n,k}(\hat{\alpha}, \hat{u})$
		\item $L(x) = (x - \alpha_{1})(x - \alpha_{2})...(x - \alpha_{n})$
		\item $L_{i}(x) = \frac{L(x)}{(x - \alpha_{i})}$
	\end{itemize}
\end{frame}

\begin{frame}
	\frametitle{Finding the Dual Code}
	\begin{itemize}
		\item $\hat{u} = (L_{1}(\alpha_{1}), L_{2}(\alpha_{2}), ... L_{n}(\alpha_{n}),) $
		\item $\hat{u} = (L_{1}(A), L_{2}(B), ... L_{23}(W)) $
		\item $\hat{u} = \{\text{2}, \text{D}, \text{V}, +, \text{9}, \text{O}, ], \text{G}, *, \wedge, \text{3}, \text{5}, \text{X}, \text{,}, \text{A}, \text{A}, >, <, \text{C}, \text{8}, \text{G}, \text{E}, \text{:}\} $
	\end{itemize}
\end{frame}

\begin{frame}
	\frametitle{Finding the Syndrome Polynomial}
	\begin{itemize}
	\item
	\end{itemize}
\end{frame}


\section{Conclusion}

\begin{frame}
	\frametitle{Final Thoughts}
\end{frame}

\begin{frame}
	\frametitle{Questions}
\end{frame}

\end{document}