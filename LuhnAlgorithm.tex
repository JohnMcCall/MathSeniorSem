\documentclass{article}
\usepackage{amsmath}

\title{Luhn Algorithm}
\author{John McCall}

\begin{document}

\maketitle

\section{Backround}
The Luhn algorithm, which is also called the modulus 10 algorithm, implements a simple
checksum formula and is used in validating identification numbers. Credit card numbers, IMEI
numbers, National Provider Identifier numbers, and Canadian Social Insurance Numbers are a
few examples of the types of ids that the Luhn algorithm is used with. It was created by
Hans Peter Luhn, a scientist working for IBM, in 1954.

This algorithm is meant to detect accidental errors, as opposed to malicious attacks, so by
design it is not a cryptographically secure function. It is mostly used as an easy way to
distinguish valid numbers from a series of random digits.

\section{The Algorithm}

\end{document}



